\documentclass{beamer}
\usetheme{Madrid}
\usecolortheme{seahorse}

% Language setting
\usepackage[main=brazilian, english]{babel}

\usepackage{tcolorbox}
\usepackage{listings}
\usepackage{tikz}

\setbeamerfont{title}{series=\bfseries}
\setbeamerfont{subtitle}{series=\normalfont}
\defbeamertemplate*{title page}{customized}[1][]
{
    \vbox{}
    \vfill
    \begingroup
        \centering
        \usebeamertemplate{titlegraphic}
        \inserttitlegraphic
        \par
        
        \vskip 0.15cm

        \usebeamerfont{institute}
        \insertinstitute
        \par

        \vskip 0.75cm

        \begin{tcolorbox}[colframe=darkgray, colback=lightgray]

            \centering
            \usebeamerfont{title}
            \inserttitle
            \par

            \usebeamerfont{subtitle}
            \insertsubtitle
            \par

        \end{tcolorbox}

        \vskip 0.75cm

        \usebeamerfont{author}
        \insertauthor
        \par
        \usebeamerfont{author}
        Turma U
        \par

        \vskip 0.75cm

        \usebeamerfont{date}
        \insertdate
        \par

    \endgroup
    \vfill
}

\titlegraphic
{
    \includegraphics[width=0.2\linewidth]{images/logos/UFRGS.png}
    \hspace{0.5cm}
    \includegraphics[width=0.2\linewidth]{images/logos/INF.png}        
}
\title[Projeto PC-PO/HLS]{Projeto PC-PO/HLS}
\subtitle{Operações com Matrizes 2x2}
\author[Pedro Lubaszewski Lima]{Pedro Lubaszewski Lima (00341810)}
\institute[]{INF01175\\Sistemas Digitais para Computadores A}
\date[\today]{\today}

\renewcommand{\lstlistingname}{Código}

\begin{document}

    \begin{frame}
        \maketitle
    \end{frame}

    \begin{frame}
        \frametitle{Sumário}
        \tableofcontents
    \end{frame}

    \section{Enunciado do Problema}
    \begin{frame}
        \frametitle{Enunciado do Problema}

        \begin{block}<1->
            {Objetivo}
            Projetar e descrever em VHDL um
            circuito que multiplique duas matrizes,
            some com uma terceira matriz e filtre a
            matriz resultado de acordo com a definição
            do cálculo abaixo. Além disso, compararar
            a construção manual com a solução HLS da
            Vitis.
        \end{block}

        \begin{block}<2->
            {Função da Saída do Sistema $ (R_{2x2}) $}
            Dadas as matrizes $ A_{2x2} $, $ B_{2x2} $
            e $ C_{2x2} $ tais que $ a_{ij} $, $ b_{ij} $,
            $ c_{ij} \in \mathbb{N} $ e $ a_{ij} < 255 $,
            $ b_{ij} < 255 $, $ c_{ij} < 65535 $, $ \forall i $,
            $ j \in \{ 1,2 \}$, e 

            $$ F(M) = 
                \begin{cases}
                    m_{ij} & \text{se } 0 < m_{ij} \leqslant 128 \\
                    128 & \text{se } m_{ij} > 128
                \end{cases}
                \text{ , então}
            $$

            $$ R = F[(A \times B) + C] $$
        \end{block}
    \end{frame}

    \section{Resolução PC-PO}
    \begin{frame}
        \frametitle{Fluxograma ASM}
    \end{frame}

    \section{Resolução HLS}
    \subsection{Programa em C}
    \begin{frame}
        \frametitle{Algoritmo em C}

        \lstinputlisting
        [caption= \textit{Header} de \texttt{matrix\_operations.cpp},
        captionpos=b,
        basicstyle=\tiny,
        numbers=left,
        columns=flexible,
        xleftmargin=26pt,
        frame=single,
        framexleftmargin=22pt,
        language=C++]
        {./code/matrix_operations.h}
    \end{frame}

    \begin{frame}
        \frametitle{Algoritmo em C}

        \lstinputlisting
        [caption= Implementação de \texttt{matrix\_operations.h},
        captionpos=b,
        basicstyle=\tiny,
        numbers=left,
        columns=flexible,
        xleftmargin=26pt,
        frame=single,
        framexleftmargin=22pt,
        language=C++]
        {./code/matrix_operations.cpp}
    \end{frame}

\end{document}