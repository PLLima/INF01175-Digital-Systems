\documentclass{beamer}
\usetheme{Madrid}
\usecolortheme{seahorse}
\setbeamertemplate{caption}[numbered]

\usepackage{lmodern}

% Language setting
\usepackage[main=brazilian, english]{babel}

\usepackage{tcolorbox}
\usepackage{listings}
\usepackage{adjustbox}

\usepackage{caption}

\setbeamerfont{title}{series=\bfseries}
\setbeamerfont{subtitle}{series=\normalfont}
\defbeamertemplate*{title page}{customized}[1][]
{
    \vbox{}
    \vfill
    \begingroup
        \centering
        \usebeamertemplate{titlegraphic}
        \inserttitlegraphic
        \par
        
        \vskip 0.15cm

        \usebeamerfont{institute}
        \insertinstitute
        \par

        \vskip 0.75cm

        \begin{tcolorbox}[colframe=darkgray, colback=lightgray]

            \centering
            \usebeamerfont{title}
            \inserttitle
            \par

            \usebeamerfont{subtitle}
            \insertsubtitle
            \par

        \end{tcolorbox}

        \vskip 0.75cm

        \usebeamerfont{author}
        \insertauthor
        \par
        \usebeamerfont{author}
        Turma U
        \par

        \vskip 0.75cm

        \usebeamerfont{date}
        \insertdate
        \par

    \endgroup
    \vfill
}

\AtBeginSection{
    \begin{frame}
        \frametitle{Sumário}
        \tableofcontents[currentsection]
    \end{frame}
}

\titlegraphic
{
    \includegraphics[width=0.2\linewidth]{images/logos/UFRGS.png}
    \hspace{0.5cm}
    \includegraphics[width=0.2\linewidth]{images/logos/INF.png}        
}
\title[Projeto PC-PO/HLS]{Projeto PC-PO/HLS}
\subtitle{Operações com Matrizes 2x2}
\author[Pedro Lubaszewski Lima]{Pedro Lubaszewski Lima (00341810)}
\institute[]{INF01175\\Sistemas Digitais para Computadores A}
\date[\today]{\today}

\renewcommand{\lstlistingname}{Código}

\begin{document}

    \begin{frame}
        \maketitle
    \end{frame}

    \begin{frame}
        \frametitle{Sumário}
        \tableofcontents
    \end{frame}

    \section{Enunciado do Problema}
    \begin{frame}
        \frametitle{Enunciado do Problema}

        \begin{block}<1->
            {Objetivo}
            Projetar e descrever em VHDL um
            circuito que multiplique duas matrizes,
            some com uma terceira matriz e filtre a
            matriz resultado de acordo com a definição
            do cálculo abaixo. Além disso, comparar
            a construção manual com a solução HLS do
            Xilinx Vitis HLS.
        \end{block}

        \begin{block}<2->
            {Função da Saída do Sistema $ (R_{2x2}) $}
            Dadas as matrizes $ A_{2x2} $, $ B_{2x2} $
            e $ C_{2x2} $ tais que $ a_{ij} $, $ b_{ij} $,
            $ c_{ij} \in \mathbb{N} $ e $ a_{ij} < 255 $,
            $ b_{ij} < 255 $, $ c_{ij} < 65535 $, $ \forall i $,
            $ j \in \{ 1,2 \}$, e $ F(M_{2x2}) = Q_{2x2} $ tal que

            $$ q_{ij} = 
                \begin{cases}
                    m_{ij} & \text{se } 0 < m_{ij} \leqslant 128 \\
                    128 & \text{se } m_{ij} > 128
                \end{cases}
                \text{,} \,\forall i \text{, } j \in \{ 1,2 \} \text{, então}
            $$

            $$ R = F[(A \times B) + C] $$
        \end{block}
    \end{frame}

    \section{Resolução PC-PO}
    \subsection{Fluxograma ASM}
    \begin{frame}
        \frametitle{Fluxograma ASM}

        \begin{onlyenv}<1>
            \begin{figure}[h]
                \centering
                \adjincludegraphics[height=6.8cm, trim={0 {0.5\height} 0 0}, clip]{./images/solutions/PC-PO_-_ASM.jpg}
                \caption{\label{img:asm-1} Fluxograma ASM – Parte 1}
            \end{figure}
        \end{onlyenv}

        \begin{onlyenv}<2>
          \begin{figure}[h]
            \centering
            \adjincludegraphics[height=6.8cm, trim={0 0 0 {0.5\height}}, clip]{./images/solutions/PC-PO_-_ASM.jpg}
            \caption{\label{img:asm-2} Fluxograma ASM – Parte 2}
          \end{figure}
        \end{onlyenv}
    \end{frame}

    \subsection{Parte Operativa}
    \begin{frame}
        \frametitle{\textit{Datapath}}
        \begin{onlyenv}<1>
            \begin{figure}[h]
                \centering
                \adjincludegraphics[height=6.8cm, trim={0 {0.593\height} 0 0}, clip]{./images/solutions/PC-PO_-_PO.jpg}
                \caption{\label{img:po-1} Parte Operativa – Parte 1}
            \end{figure}
        \end{onlyenv}

        \begin{onlyenv}<2>
          \begin{figure}[h]
            \centering
            \adjincludegraphics[height=6.8cm, trim={0 0 0 {0.407\height}}, clip]{./images/solutions/PC-PO_-_PO.jpg}
            \caption{\label{img:po-2} Parte Operativa – Parte 2}
          \end{figure}
        \end{onlyenv}
    \end{frame}

    \subsection{Parte de Controle}
    \begin{frame}
        \frametitle{FSM}
        \begin{onlyenv}<1>
            \begin{figure}[h]
                \centering
                \adjincludegraphics[height=6.8cm, trim={0 {0.53\height} 0 0}, clip]{./images/solutions/PC-PO_-_FSM.jpg}
                \caption{\label{img:fsm-1} Parte de Controle – Parte 1}
            \end{figure}
        \end{onlyenv}

        \begin{onlyenv}<2>
          \begin{figure}[h]
            \centering
            \adjincludegraphics[height=6.8cm, trim={0 0 0 {0.47\height}}, clip]{./images/solutions/PC-PO_-_FSM.jpg}
            \caption{\label{img:fsm-2} Parte de Controle – Parte 2}
          \end{figure}
        \end{onlyenv}
    \end{frame}

    \subsection{Validações da Solução}
    \begin{frame}
        \frametitle{Propostas de Testes}

        \begingroup \onslide<1->
        Para testar a solução, foram criadas as seguintes entradas:

        \begin{itemize}
            \item $ A = \begin{bmatrix} 1 & 2 \\ 3 & 4 \end{bmatrix} $;
            \item $ B = \begin{bmatrix} 4 & 3 \\ 2 & 1 \end{bmatrix} $;
            \item $ B_{NULL} = \begin{bmatrix} 0 & 0 \\ 0 & 0 \end{bmatrix} $;
            \item $ C = \begin{bmatrix} 121 & 10 \\ 50 & 3 \end{bmatrix} $.
        \end{itemize}
        \endgroup

        \begingroup \onslide<2->
        Esperando as seguintes saídas:
        \begin{itemize} \onslide<2->
            \item $ R_1 = \begin{bmatrix} 128 & 15 \\ 70 & 16 \end{bmatrix} $,
                  com $ A $, $ B $ e $ C $ como entradas;
            \item $ R_2 = \begin{bmatrix} 121 & 10 \\ 50 & 3 \end{bmatrix} $,
                  com $ A $, $ B_{NULL} $ e $ C $ como entradas.
        \end{itemize}
        \endgroup

    \end{frame}

    \begin{frame}
        \frametitle{Simulação para Validação de $ R_1 $}

        \begingroup \onslide<2->
            Como a saída $ doutr = [128, 15, 70, 16] $, 
            no modo da memória \textit{read first}, então
            foi validado que $ R_1 = \begin{bmatrix} 128 & 15 \\ 70 & 16 \end{bmatrix} $.
        \endgroup

        \begin{figure}[h] \onslide<1->
            \centering
            \adjincludegraphics[width=0.85\linewidth, trim={{0.4\height} {0.3\height} 0 0}, clip]{./images/solutions/PC-PO-Sim1.png}
            \caption{\label{img:pc-po-simulation-1} Forma de Onda Funcional com entradas $ A $, $ B $ e $ C $}
          \end{figure}
    \end{frame}

    \begin{frame}
        \frametitle{Simulação para Validação de $ R_2 $}

        \begingroup \onslide<2->
            Como a saída $ doutr = [121, 10, 50, 3] $, 
            no modo da memória \textit{read first}, então
            foi validado que $ R_2 = \begin{bmatrix} 121 & 10 \\ 50 & 3 \end{bmatrix} $.
        \endgroup

        \begin{figure}[h] \onslide<1->
            \centering
            \adjincludegraphics[width=0.85\linewidth, trim={{0.4\height} {0.3\height} 0 0}, clip]{./images/solutions/PC-PO-Sim2.png}
            \caption{\label{img:pc-po-simulation-2} Forma de Onda Funcional com entradas $ A $, $ B_{NULL} $ e $ C $}
          \end{figure}
    \end{frame}

    \section{Resolução HLS}
    \subsection{Programa em C}
    \begin{frame}[fragile]
        \frametitle{Algoritmo em C}

        \lstinputlisting
        [caption= \textit{Header} de \texttt{matrix\_operations.cpp},
        captionpos=b,
        basicstyle=\tiny,
        numbers=left,
        columns=flexible,
        xleftmargin=26pt,
        frame=single,
        framexleftmargin=22pt,
        language=C++]
        {./code/matrix_operations.h}
    \end{frame}

    \begin{frame}[fragile]
        \frametitle{Algoritmo em C}

        \lstinputlisting
        [caption= Implementação de \texttt{matrix\_operations.h},
        captionpos=b, 
        basicstyle=\tiny,
        numbers=left,
        columns=flexible,
        xleftmargin=26pt,
        frame=single,
        framexleftmargin=22pt,
        language=C++]
        {./code/matrix_operations.cpp}
    \end{frame}

    \subsection{Solução Básica com HLS}
    \begin{frame}
        \frametitle{Síntese da Solução Básica com HLS}

        \begingroup \onslide<1->
        Como solução básica,
        não se utilizou \textbf{nenhuma otimização}.
        \endgroup

        \begin{figure}[h] \onslide<2->
            \centering
            \includegraphics[width=1\linewidth]{./images/solutions/HLS_-_Basic_Solution.png}
            \caption{\label{img:syn-hls-basic-solution} Resultado da Síntese Básica com HLS}
          \end{figure}
    \end{frame}

    \begin{frame}
        \frametitle{Interface da Solução Básica com HLS}

        \begingroup \onslide<2->
        Percebe-se um total de \textbf{12 IOBs}, totalizando
        \textbf{68 \textit{bits}} de interface de dados.
        \endgroup

        \begin{figure}[h] \onslide<1->
            \centering
            \includegraphics[width=0.62\linewidth]{./images/solutions/VHDL_-_Basic_Solution.png}
            \caption{\label{img:int-hls-basic-solution} Interface da Solução Básica com HLS}
          \end{figure}
    \end{frame}

    \subsection{Primeira Solução Otimizada com HLS}
    \begin{frame}
        \frametitle{Síntese da Primeira Solução Otimizada com HLS}

        \begingroup \onslide<1->
        Como primeira solução otimizada, utilizou-se \textbf{\texttt{loop unroll}}
        em cada um dos \textit{loops} da aplicação, buscando aumentar o paralelismo
        da solução.
        \endgroup

        \begin{figure}[h] \onslide<2->
            \centering
            \includegraphics[width=1\linewidth]{./images/solutions/HLS_-_Loop_Unroll.png}
            \caption{\label{img:syn-hls-optimized-solution-1} Resultado da Síntese da Solução Otimizada 1 com HLS}
          \end{figure}
    \end{frame}

    \begin{frame}
        \frametitle{Interface da Primeira Solução Otimizada com HLS}

        \begingroup \onslide<2->
        Percebe-se um total de \textbf{16 IOBs}, totalizando
        \textbf{96 bits} de interface de dados.
        \endgroup

        \begin{figure}[h] \onslide<1->
            \centering
            \includegraphics[width=0.55\linewidth]{./images/solutions/VHDL_-_Loop_Unroll.png}
            \caption{\label{img:int-hls-optimized-solution-1} Interface da Solução Otimizada 1 com HLS}
          \end{figure}
    \end{frame}

    \subsection{Segunda Solução Otimizada com HLS}
    \begin{frame}
        \frametitle{Síntese da Segunda Solução Otimizada com HLS}

        \begingroup \onslide<1->
        Além das otimizações da versão anterior (\texttt{loop unroll}), acrescentou-se
        \textbf{\texttt{array partitioning}} com dimensão zero em todas as entradas e
        saídas do sistema. Isso foi feito em busca do maior nível de paralelismo possível
        para este circuito.
        \endgroup

        \begin{figure}[h] \onslide<2->
            \centering
            \includegraphics[width=1\linewidth]{./images/solutions/HLS_-_Unroll_and_Array_Partitioning.png}
            \caption{\label{img:syn-hls-optimized-solution-2} Resultado da Síntese da Solução Otimizada 2 com HLS}
          \end{figure}
    \end{frame}

    \begin{frame}
        \frametitle{Interface da Segunda Solução Otimizada com HLS}

        \begingroup \onslide<2->
        Percebe-se um total de \textbf{16 IOBs}, totalizando
        \textbf{160 bits} de interface de dados.
        \endgroup

        \begin{figure}[h] \onslide<1->
            \centering
            \includegraphics[width=0.55\linewidth]{./images/solutions/VHDL_-_Unroll_and_Array_Partitioning.png}
            \caption{\label{img:int-hls-optimized-solution-2} Interface da Solução Otimizada 2 com HLS}
          \end{figure}
    \end{frame}

    \section{Comparação das Soluções}
    \begin{frame}
        \frametitle{Comparação do PC-PO com HLS}
        \begin{columns}
            \onslide<1->
            \begin{column}{0.49\textwidth}
                Ao construir a solução partindo do \textbf{Fluxograma ASM},
                da \textbf{Parte Operativa} e da \textbf{Parte de Controle},
                obteve-se um circuito com:

                \begin{itemize}
                    \item \textcolor<3->{green}{174 LUTs};
                    \item \textcolor<3->{red}{43 flip flops};
                    \item \textcolor<3->{green}{0 DSPs};
                    \item \textcolor<3->{red}{4 BRAMs};
                    \item Latência de \textcolor<3->{red}{76 ciclos de \textit{clock}};
                    \item Interface de dados com \textcolor<3->{green}{4 IOBs, 40 bits}.
                \end{itemize}

            \end{column}

            \onslide<2->
            \begin{column}{0.02\textwidth}
                \rule{.1mm}{0.65\textheight}
            \end{column}
            \begin{column}{0.49\textwidth}
                Através da melhor solução com \textbf{HLS}, obteve-se os
                seguintes resultados:

                \begin{itemize}
                    \item \textcolor<3->{red}{405 LUTs};
                    \item \textcolor<3->{green}{36 flip flops};
                    \item \textcolor<3->{red}{4 DSPs};
                    \item \textcolor<3->{green}{0 BRAMs};
                    \item Latência de \textcolor<3->{green}{3 ciclos de \textit{clock}};
                    \item Interface de dados com \textcolor<3->{red}{16 IOBs, 160 bits}.
                \end{itemize}

            \end{column}
        \end{columns}

        \onslide<3->
        {
            De forma geral, a solução HLS tem um \textbf{paralelismo} melhor que a PC-PO.
            No entanto, a solução PC-PO apresenta uma \textbf{melhor utilização de recursos} (LUTs
            em particular) e uma \textbf{interface mais compacta}.
        }

    \end{frame}

    \begin{frame}
        \centering \huge
        \vfill
        Obrigado pela atenção!

        Alguma dúvida?
        \vfill
    \end{frame}

\end{document}